\section{Conclusion}

In this work, we studied the impact of different levels of haptic feedback fidelity on motor learning, by systematically reviewing current research based on the Haptic Feedback Fidelity Framework, which we extended to include measures of confidence and motor learning. 
Our findings highlight the importance of applying a common framework for evaluating feedback fidelity. We found that low feedback fidelity, often resulting from design flaws or a mismatch between the feedback and the task, generally leads to poor motor performance. Conversely, medium and high-fidelity feedback, aligning closely with the natural forces of the natural occurrence of the task, can greatly improve motor learning. 

We hope that this study serves as a stepping stone towards a common understanding of haptic feedback fidelity to foster our understanding of its impact on motor learning, and to ultimately make (re-)learning new skills in VR more effective. 