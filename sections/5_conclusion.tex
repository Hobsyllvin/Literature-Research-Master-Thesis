\section{Conclusion}

In this work, we studied the impact of different levels of haptic feedback fidelity on motor learning, by systematically reviewing current research based on the Haptic Fidelity Framework, which we extended to include measures of confidence and motor learning. 
We found that low feedback fidelity, often resulting from design flaws or a mismatch between the feedback and the task, generally leads to poorer motor performance. Conversely, medium and high-fidelity feedback, aligning closely with the forces involved in the natural occurrence of the task, can greatly improve motor learning. 
Furthermore, our findings emphasize the importance of applying a common framework when evaluating feedback fidelity to ensure comparability of results across studies on the impact of haptic feedback.

We hope that this study serves as a stepping stone towards a common understanding of haptic feedback fidelity to foster our understanding of its impact on motor learning, and to ultimately make (re-)learning new skills in VR more effective. 