%%%%% RESULTS %%%%%%
\twocolumn

\section{Results}
\label{sec:results}
Figure \ref{fig:fidelity_plot} displays the haptic feedback fidelity and versatility scores of the systems described in each article. The quality score of each system is depicted by the dot's color and size, respectively.

\begin{figure*}[htbp]
    \begin{tikzpicture}[scale=3.5]
    
    % Add axis labels
    \foreach \x in {0,1,2,3,4} {
        \draw [very thin, lightgray](\x cm, 0-0.05) -- (\x cm, 4+0.05) node[anchor=north] {};
        \draw [very thin, lightgray](0-0.05,\x cm) -- (4+0.05,\x cm) node[anchor=east] {};
    }

    % Draw axes
    \draw[thick,<->] (0,2) -- (4,2) node[anchor=south west] {Haptic Fidelity};
    \draw[thick,<->] (2,0) -- (2,4) node[anchor=south] {Versatility};

    \node at (0,1.9) {\footnotesize{abstract}};
    \node at (4,1.9) {\footnotesize{realistic}};
    \node at (1.8,0.1) {\footnotesize{specific}};
    \node at (1.8,3.9) {\footnotesize{generic}};

    % Legend
    \draw[fill=white] (0.1,3.9) rectangle (1.2,3.25); % Legend border
    \node[anchor=west] at (0.15, 3.8) {\textbf{Legend}}; % Legend title
    
    \node[circle, fill=c1, inner sep=2.3pt] at (0.22, 3.65) {};
    \node[anchor=west] at (0.25, 3.65) {\footnotesize{Quality q > 0.9}};

    \node[circle, fill=c2, inner sep=2pt] at (0.22, 3.55) {};
    \node[anchor=west] at (0.25, 3.55) {\footnotesize{Quality 0.8 < q \leq 0.9}};

    \node[circle, fill=c3, inner sep=1.7pt] at (0.22, 3.45) {};
    \node[anchor=west] at (0.25, 3.45) {\footnotesize{Quality 0.7 < q \leq 0.8}};

    \node[circle, fill=c4, inner sep=1.4pt] at (0.22, 3.35) {};
    \node[anchor=west] at (0.25, 3.35) {\footnotesize{Quality 0.6 q \leq 0.6}};
    
    % Sample data points
    \node[circle, fill=c1, inner sep=2.3pt] at (3.5,2) {};
    \node[circle, fill=c1, inner sep=2.3pt] at (3.22,2) {};
    \node[circle, fill=c1, inner sep=2.3pt] at (0.63,2) {};
    \node[circle, fill=c1, inner sep=2.3pt] at (3.38,2) {};
    \node[circle, fill=c1, inner sep=2.3pt] at (3.75,2) {};
    \node[circle, fill=c1, inner sep=2.3pt] at (2.63,2) {};
    \node[circle, fill=c1, inner sep=2.3pt] at (2.75,2) {};
    \node[circle, fill=c1, inner sep=2.3pt] at (3.21,0) {};
    \node[circle, fill=c1, inner sep=2.3pt] at (4,0) {};
    \node[circle, fill=c1, inner sep=2.3pt] at (4,1) {};
    \node[circle, fill=c1, inner sep=2.3pt] at (3.88,1) {};
    \node[circle, fill=c1, inner sep=2.3pt] at (1.5,3) {};
    \node[circle, fill=c1, inner sep=2.3pt] at (3.59,1) {};
    \node[circle, fill=c1, inner sep=2.3pt] at (3.09,2) {};
    \node[circle, fill=c1, inner sep=2.3pt] at (2.57,4) {};
    \node[circle, fill=c1, inner sep=2.3pt] at (3.21,1) {};
    \node[circle, fill=c1, inner sep=2.3pt] at (3.01,2) {};
    \node[circle, fill=c1, inner sep=2.3pt] at (3.25,2) {};
    \node[circle, fill=c1, inner sep=2.3pt] at (2.84,3) {};
    \node[circle, fill=c1, inner sep=2.3pt] at (3.7,1) {};
    \node[circle, fill=c1, inner sep=2.3pt] at (2.57,4) {};
    \node[circle, fill=c1, inner sep=2.3pt] at (3.5,2) {};
    \node[circle, fill=c2, inner sep=2.0pt] at (2.63,4) {};
    \node[circle, fill=c2, inner sep=2.0pt] at (2.84,2) {};
    \node[circle, fill=c2, inner sep=2.0pt] at (3.27,2) {};
    \node[circle, fill=c2, inner sep=2.0pt] at (3.34,1) {};
    \node[circle, fill=c2, inner sep=2.0pt] at (3.46,1) {};
    \node[circle, fill=c2, inner sep=2.0pt] at (3.46,1) {};
    \node[circle, fill=c2, inner sep=2.0pt] at (3.46,1) {};
    \node[circle, fill=c2, inner sep=2.0pt] at (3.62,1) {};
    \node[circle, fill=c2, inner sep=2.0pt] at (3.75,2) {};
    \node[circle, fill=c2, inner sep=2.0pt] at (3.09,1) {};
    \node[circle, fill=c2, inner sep=2.0pt] at (3.71,1) {};
    \node[circle, fill=c2, inner sep=2.0pt] at (3.29,2) {};
    \node[circle, fill=c2, inner sep=2.0pt] at (2.28,1) {};
    \node[circle, fill=c2, inner sep=2.0pt] at (3.42,3) {};
    \node[circle, fill=c2, inner sep=2.0pt] at (2.76,1) {};
    \node[circle, fill=c2, inner sep=2.0pt] at (3.71,2) {};
    \node[circle, fill=c3, inner sep=1.7pt] at (2.2,2) {};
    \node[circle, fill=c3, inner sep=1.7pt] at (2.8,2) {};
    \node[circle, fill=c3, inner sep=1.7pt] at (1.25,3) {};
    \node[circle, fill=c3, inner sep=1.7pt] at (1.39,2) {};
    \node[circle, fill=c3, inner sep=1.7pt] at (2.49,3) {};
    \node[circle, fill=c3, inner sep=1.7pt] at (3.61,0) {};
    \node[circle, fill=c3, inner sep=1.7pt] at (1.08,1) {};
    \node[circle, fill=c3, inner sep=1.7pt] at (2.31,3) {};
    \node[circle, fill=c4, inner sep=1.4pt] at (4,1) {};


    % Add citations to datapoints
    \node at (3.5,2.1) {\footnotesize{\cite{Brickler2019}}};
    \node at (3.22,2.1) {\footnotesize{\cite{Caccianiga2021}}};
    \node at (0.63,2.1) {\footnotesize{\cite{Crespo2015}}};
    \node at (3.38,1.9) {\footnotesize{\cite{Feygin2002HapticSkill}}};
    \node at (3.75,2.2) {\footnotesize{\cite{Feygin2002HapticSkill}}};
    \node at (2.63,2.1) {\footnotesize{\cite{Gambaro2014}}};
    \node at (2.75,1.9) {\footnotesize{\cite{Gambaro2014}}};
    \node at (3.21,0.1) {\footnotesize{\cite{Graham2008}}};
    \node at (4,0.1) {\footnotesize{\cite{Huang2006}}};
    \node at (4,1.1) {\footnotesize{\cite{Huang2007}}};
    \node at (3.88,1.1) {\footnotesize{\cite{LeeH2014}}};
    \node at (1.5,3.1) {\footnotesize{\cite{LiuH2019}}};
    \node at (3.59,1.1) {\footnotesize{\cite{Mohanty2023}}};
    \node at (3.09,1.9) {\footnotesize{\cite{Oquendo2024}}};
    \node at (2.57,3.9) {\footnotesize{\cite{Vasudevan2020}}};
    \node at (3.21,1.1) {\footnotesize{\cite{Dai2023}}};
    \node at (3.01,2.1) {\footnotesize{\cite{Dai2023}}};
    \node at (3.25,1.9) {\footnotesize{\cite{Gunter2022}}};
    \node at (2.84,3.1) {\footnotesize{\cite{LeeY2019}}};
    \node at (3.7,1.1) {\footnotesize{\cite{LiuG2014}}};
    \node at (2.57,4.1) {\footnotesize{\cite{McAnally2023}}};
    \node at (3.5,2.2) {\footnotesize{\cite{Rodriguez2010}}};
    \node at (2.7,4.1) {\footnotesize{\cite{Yang2023}}};
    \node at (2.84,2.2) {\footnotesize{\cite{Yang2023}}};
    \node at (3.27,2.2) {\footnotesize{\cite{Yang2023}}};
    \node at (3.34,1.1) {\footnotesize{\cite{Fehlberg2012}}};
    \node at (3.39,0.9) {\footnotesize{\cite{Fehlberg2012}}};
    \node at (3.51,0.9) {\footnotesize{\cite{Fehlberg2012}}};
    \node at (3.46,1.1) {\footnotesize{\cite{Fehlberg2012}}};
    \node at (3.62,0.9) {\footnotesize{\cite{Fehlberg2012}}};
    \node at (3.75,2.1) {\footnotesize{\cite{Fehlberg2012}}};
    \node at (3.09,1.1) {\footnotesize{\cite{Grant2019}}};
    \node at (3.71,1.2) {\footnotesize{\cite{Macuga2019}}};
    \node at (3.35,2.1) {\footnotesize{\cite{Morris2007}}};
    \node at (2.28,1.1) {\footnotesize{\cite{Najdovski2020}}};
    \node at (3.42,3.1) {\footnotesize{\cite{Oezen2022}}};
    \node at (2.76,1.1) {\footnotesize{\cite{Vaghela2021}}};
    \node at (3.71,1.9) {\footnotesize{\cite{Wall2000}}};
    \node at (2.2,2.1) {\footnotesize{\cite{Chappell2022}}};
    \node at (2.8,2.1) {\footnotesize{\cite{Chi2017}}};
    \node at (1.25,3.1) {\footnotesize{\cite{Hanashima2023}}};
    \node at (1.39,2.1) {\footnotesize{\cite{Perez2023}}};
    \node at (2.49,3.1) {\footnotesize{\cite{Trinitatova2023}}};
    \node at (3.61,0.1) {\footnotesize{\cite{Vaghela2021}}};
    \node at (1.08,1.1) {\footnotesize{\cite{Lee2012}}};
    \node at (2.31,3.1) {\footnotesize{\cite{Xia2023}}};
    \node at (4,0.9) {\footnotesize{\cite{Manivannan2008}}};

    
\end{tikzpicture}
    \caption{Haptic fidelity and versatility scores for the included papers}
    \label{fig:fidelity_plot}
\end{figure*}

\subsection{K-means clustering}
\begin{itemize}
    \item Explain k-means clustering and silhouette score
\end{itemize}

In order to structure the data, we clustered the evaluated systems using the k-means algorithm. The optimal number of clusters was chosen based on the highest \textit{silhouette score} for $n<10$ (see figure \ref{fig:silhouette}), which is calculated the following \cite{Rousseeuw1987Silhouettes:Analysis}:
\begin{enumerate}
    \item Calculate the \textit{silhouette coefficient} for all instances equal to $(b-a) / max(a,b)$, where $a$ is the mean distance to the other instances in the same cluster, and $b$ is the mean distance to the instances of the next closest cluster.
    \item Compute the mean \textit{silhouette coefficient} over all instances, which equals the \textit{silhouette score}.
\end{enumerate}

\begin{figure}[htbp]
    \centering
    \includegraphics[width=\columnwidth]{figures/silhouette.pdf} 
    \caption{Silhouette scores for the numbers of clusters}
    \label{fig:silhouette}
\end{figure}

As can be seen in figure \ref{fig:kmeans}, the data was separated in 9 different clusters. Each cluster center is depicted by a small blue dot.

\begin{figure}[htbp]
    \centering
    \includegraphics[width=\columnwidth]{figures/literature_data.pdf} 
    \caption{KMeans clustering of the data}
    \label{fig:kmeans}
\end{figure}


\subsection{Evaluation of the clusters}
\begin{itemize}
    \item Explain each cluster, describe differences and similarities
\end{itemize}
\begin{comment}
Cluster I unites studies that have a relatively low haptic fidelity score, however they have a fairly high versatility. This is because ..., in the papers haptic feedback had this effect, ...

Cluster IV and V concern studies with ...
\end{comment}


