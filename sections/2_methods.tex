%%%%% METHODS %%%%%

\begin{figure*}[htbp]
    \centering
    \includesvg[width=\linewidth]{figures/prisma_overview.svg} 
    \caption{Overview of the methodology using the PRISMA method}
    \label{fig:prisma}
\end{figure*}

\section{Methods}
\label{sec:methods}
This systematic review is conducted utilizing the PRISMA methodology \cite{Page2021TheReviews} to ensure transparent and complete reporting of the topics examined.

\subsection{Search Strategy}

\subsubsection{Search Terms}
Several relevant articles were reviewed at the beginning to find eligible search terms. The query based on the research question consists of three main parts, namely \textit{virtual reality}, \textit{somatosensory feedback}, and \textit{motor learning}, each term with their respective synonyms. 

The final search query was as follows: motor AND (learning OR control OR training OR skills) AND (((virtual OR augmented) AND reality) OR ((remote OR virtual OR simulated) AND environment)) AND (((somatosensory OR haptic OR tactile OR proprioceptive OR kinesthetic OR cutaneous OR somatic) AND 
(cue* OR feedback OR rendering OR stimul*)) AND (fidelity OR realism OR accuracy OR precision OR exactness OR specificity)). The search terms were applied for the article title, abstract, and keywords.

Three databases (Scopus, IEEE Xplore, and PubMed) were searched in April 2024. Based on the required syntax of each database, the search query was slightly adapted. For example, as PubMed only allows for the use of an asterisk for words containing more than three letters, the term \textit{cue*} was changed to (\textit{cue} OR \textit{cues}). The exact search queries can be found in Appendix \ref{sec:queries}. 

The results were saved in the library of Mendeley (v2.112.0) by Elsevier, NL, in April 2024, which was also used for the automatic removal of duplicates. 

\subsubsection{Eligibility Criteria}
\label{sec:eligibility}
To qualify for inclusion, studies had to focus on the impact of somatosensory feedback on motor learning in humans. We included studies with healthy participants only, as motor-impaired patients often know how to perform a movement but are physically constrained by their condition, therefore augmented feedback may help the motor learning of patients in a different way compared to healthy subjects \cite{Sigrist2013AugmentedReview}. There were no publication date or language limitations.

Conversely, studies were excluded if they (i) did not relate to motor learning in humans; (ii) focused on systems not providing haptic feedback in VR; (iii) were concerned with a neurological condition affecting motor learning; or (iv) explained a new assessment system for the evaluation of motor learning.

\subsubsection{Study Selection}
At first, duplicate publications were identified and removed. The results were then screened based on their title and abstract. 

In this review, we wanted to emphasize more recent studies to reflect the significant impact of novel technological advances in VR on the field. Therefore, records published in the years 2022-2024 were re-screened and also included in the further process, if they were only partially fitting the eligibility criteria. Additional papers were identified using citation searching.

Finally, the full-text articles were assessed to determine their eligibility for inclusion in the review (see fig. \ref{fig:prisma}).


\subsection{Search Analysis}
\subsubsection{Haptic Feedback Fidelity Framework}


The fidelity of a system describes the objective degree of exactness in its output \cite{McMahan2011ExploringGames}. 
In the context of haptic feedback, this refers to the ability to generate accurate and realistic stimuli that align with user expectations based on real-world experiences during interactions with a virtual environment \cite{Muender2022HapticReality}. 

Several frameworks have been proposed to assess the feedback fidelity of the VE. They concentrate on either the physical attributes of the system (e.g. the FIFA framework developed by McMahan et al. \cite{McMahan2011ExploringGames}), aim to evaluate feedback fidelity through questionnaires (e.g. the haptics addition \cite{Boos2017ErweiterungHaptik} to the User Experience Questionnaire \cite{Laugwitz2008ConstructionQuestionnaire}) or focus on the effect of the system. Huang et al. for example suggested that skill transfer might be a useful paradigm to evaluate the level of fidelity, assuming that if an environment has infinite fidelity and is therefore perfectly recreating the sensations of the real environment, then there would be no performance loss when transferring a skill from the virtual to the real environment \cite{Huang2006}.

These frameworks, while addressing particular subjective and objective dimensions of haptic feedback systems, fail to capture their full diversity. This gap is addressed by Muender et al., who developed the Haptic Feedback Fidelity Framework \cite{Muender2022HapticReality}. It helps to evaluate a system's haptic feedback fidelity ranging from abstract to realistic and introduces versatility as a second axis, which allows for the differentiation between generic and specific haptic feedback systems. 

The evaluation of haptic feedback fidelity is based on eight foundational factors, describing the features of a system and their value, and six limiting factors that negatively impact the perception. Each factor is rated on a 5-point Likert scale (0-4). 

For this systematic review, we evaluate the selected articles described in section \ref{sec:methods}  based on this framework. This will help to create a dataset where the haptic fidelity of systems is evaluated using transparent and consistent methods to overcome the variations in fidelity classification found in different studies (e.g. \cite{Yang2023, Grant2019}. We extend the framework by a third axis, taking into account a confidence score for each haptic feedback system. This score determines how many of the parameters assessed in the framework have been provided by the researchers. Furthermore, single aspects of the framework are extended in an attempt to make the evaluation more transparent and quantifiable. 
To not describe every detail of the framework in this review, we would like to refer to the Haptic Feedback Fidelity Framework for more detail and examples \cite{Muender2022HapticReality}.

In an attempt to make some of the factors easier to quantify and to reduce the subjectiveness of the evaluation, we developed equation \ref{eq:diff}. It expresses the relation between properties in the virtual environment (addressed with $VR_{factor}$) and the natural occurrence of the haptic feedback ($RE_{factor}$). The relation is scaled so it fits the 5-point Likert scale.

\begin{equation}    
4 \cdot \left(1 - \left(\frac{\left|VR_{\text{factor}} - RE_{\text{factor}}\right|}{RE_{\text{factor}}}\right)\right)
\label{eq:diff}
\end{equation}



\paragraph{Haptic Fidelity}
\label{sec:foundationallimiting}

In the following, the \textit{foundational factors} are described.

\begin{itemize}
    \item \textbf{Body Location:} 
    This factor describes whether the haptic feedback on the user's body created by the system aligns with where the person would perceive the stimulus in the natural occurrence of the interaction \cite{Muender2022HapticReality}. Here, the ability to localize a point on the skin for different body parts is taken into account, giving more weight to body parts where the point localization threshold is significantly lower (e.g. hallux, upper lip, nose, cheek, forehead, and all fingers) \cite{Lederman2009HapticTutorial}. In our work, we weighted these body parts with a factor of two.
    Equation \ref{eq:diff} can help with the quantification of this factor, with $VR_{factor}$ being the number of body parts that are involved in VR, and $RE_{factor}$ being the number of body parts involved in the natural occurrence of the interaction.
    
    \item \textbf{Body Area:} 
    This factor accounts for the degree to which the same extent of the user's body surface is involved in the simulation compared to the natural occurrence of the intended haptics. The density of mechanoreceptors should be considered \cite{Muender2022HapticReality, Lederman2009HapticTutorial}.

    \item \textbf{Stimuli:}
    This factor quantifies the degree to which the same haptic receptors of the user are involved (see also equation \ref{eq:diff}). The table provided by Muender et al. can help distinguish which haptic stimuli are affected by which physical property of the handled object or the VE \cite{Muender2022HapticReality}.
    
    \item \textbf{Magnitude:}
    The magnitude describes to which degree the system can create the same strength (e.g. force) or variation (e.g. texture) of haptic stimuli compared to the natural occurrence of the intended haptics \cite{Muender2022HapticReality}. While some researchers state that high-fidelity haptic devices can typically provide up to 40N of force \cite{Grant2019}, we argue that the necessity to provide a certain magnitude of force highly depends on the stimulated body parts and the task itself. While a magnitude of 9 N, as provided by for example the Novint Falcon haptic device, might be sufficient for manipulating light objects with the fingers, it will arguably not be enough to mimic a ballistic movement with the entire arm handling heavy objects.
    If the system can fully match magnitude and variation, it would receive 4 points for this factor. If only magnitude or variation are matched, this would result in a score of 2.

    \item \textbf{Sensory Integrity:}
    This factor determines whether the medium that creates the haptic stimulus matches the perception of the natural occurrence and is in line with the intent \cite{Muender2022HapticReality}. For example, Zenner et al. use actuated flamenco fans to mimic the perception of weight by leveraging the drag created through the air resistance \cite{Zenner2019}. This can be rated with a medium degree of integrity, as it does not match the feeling of weight exactly, but it is superior to using non-actuated flamenco fans, or to providing no weight perception at all.
    
    \item \textbf{Degrees of Freedom:}
    This factor determines whether the system can deliver haptic feedback with the same degrees of freedom (DoF) as the natural occurrence of the intended haptics \cite{Muender2022HapticReality}. For instance, in the tank gunnery simulation introduced by Liu et al., haptic feedback in the two rotational axes is adequate, since the gun control interface itself operates with only two DoF \cite{LiuG2014}. Equation \ref{eq:diff} can help with the quantification of this factor.
    
    \item \textbf{Hardware Precision:}
    This factor is determined by the extent to which the system can reproduce the detail of haptic feedback compared to the natural occurrence of the intended haptics \cite{Muender2022HapticReality}. 
    It is influenced by the accuracy of the system (e.g. encoder resolution of stepper motors) and the refresh rate of the haptic stimulation and the display. According to Coles et al., the commonly accepted refresh rate to provide realistic force feedback is at around 1 kHz \cite{Coles2011TheArt}, depending on the object's stiffness. Coles et al. state that hard objects require a higher refresh rate than soft objects. Although we agree with this terminus, we suggest that also the type of task has to be taken into account when evaluating hardware precision. Fast and precise tasks such as following a 3D path with a ring in the shortest possible time \cite{Oquendo2024} require a higher feedback frequency than controlling an underwater robot through currents with moderate speed and predictable obstacles \cite{Xia2023}.
    For this review, we give a perfect score for a system with a 1,000 Hz refresh rate for the haptics regardless of the task. Depending on the task and the object stiffness we give fewer points to the system if its refresh rate is lower.

    Regarding the accuracy of the haptic feedback, the human ability to localize a point on the skin needs to be considered, as the feedback has to be more precise when applied to sensitive areas such as the fingertips compared to less sensitive areas such as the shoulder \cite{Lederman2009HapticTutorial}.
    
    Regarding the display frequency, according to Lim et al., an update rate of 30 Hz is required for the display to be realistic \cite{Lim2021}, whereas Han et al. found that there was a great increase for frequencies up to 120 Hz in motion perception \cite{Han2022AssessingPotentials}. This is especially important when objects are moving quickly, or when the visual perception of one's own body parts is important for the task execution.
    We therefore give a perfect score if the display frequency is greater or equal to 120 Hz, and reduce the score if the display frequency is lower and the task requires fast motion.
    
    \item \textbf{Software Precision:}
    This factor describes the detail to which the software can simulate the intended haptic feedback. This concerns for example the exactness of collision boxes in the software \cite{Muender2022HapticReality}.
    
\end{itemize}



In the following, the \textit{limiting factors} are described. As these factors can only negatively impact the feedback fidelity, they are rated with 0 if they have no impact, and with a score of up to 4, if the impact covers up wanted effects.

\begin{itemize}
    \item \textbf{Dependency:}
    The factor dependency measures to which degree the absence of different dependent haptic stimuli, that are usually perceived together in reality, has an impact on the haptic perception of the system. For more details see \cite{Muender2022HapticReality}.
    
    \item \textbf{Distinguishability:}
    This factor describes if different physical properties in the VE can be distinguished by the haptic feedback the system provides. If the system for example only gives vibrotactile feedback when manipulating an object, this would increase the score of this limiting factor, as different properties of the object such as weight or texture cannot be distinguished. 

    \item \textbf{Hardware Latency:}
    The delay between the interaction with the object in the VE and the haptic stimulation by the system determines this factor. While a latency of up to 25ms has been shown not to affect the user, delays greater than 25ms can have a noticeable impact on performance \cite{Muender2022HapticReality}.
    
    \item \textbf{Software Latency:}
    The same threshold as for hardware latency also applies to this factor, which is concerned with the software latency's impact on the system's haptic perception.

    \item \textbf{Side Effects:}
    Haptic systems can produce unintended haptic stimuli such as vibration or friction. This factor describes how these side effects influence the haptic perception of the system. 

    \item \textbf{Constraints:}
    This factor describes the degree to which the system constrains the user's movement other than in the intended way \cite{Muender2022HapticReality}.
    
\end{itemize}

\textbf{Scoring:}
The score for the haptic fidelity is calculated based on equation \ref{eq:fidelity_score}, with $N_F$ for the number of foundational factors, $X_F$ for the respective rating of these factors, $N_L$ for the number of limiting factors and $X_L$ for their respective rating \cite{Muender2022HapticReality}:

\begin{equation}
    \frac{\sum_{N_F}^{0} X_F}{N_F} \cdot e^{-0.0027 \left(\sum_{N_L}^{0} X_F^2 \right)^2}
    \label{eq:fidelity_score}
\end{equation}


\paragraph{Versatility}
The versatility is assessed on a 5-point Likert scale with 0 for very specific systems that can only be used for one scenario and 4 for systems that can produce very generic haptic feedback. For the classification, we follow the example categories provided by Muender et al. \cite{Muender2022HapticReality}.


\paragraph{Confidence Score}
\label{sec:confidence}
In total, 14 parameters needed to be evaluated for each haptic feedback system. We calculated a score based on the proportion of factors, that were not described in the paper itself, nor could be accessed through internet research (e.g. by looking up the specifications for the commercial products).
We divided the papers into four groups based on the quality score:

\begin{enumerate}
	\item High confidence: $C > 0.9$
	\item Good confidence: $0.8 < C \leq 0.9$
	\item Medium confidence: $0.7 < C \leq 0.8$
	\item Low confidence: $C \leq 0.7$
\end{enumerate} 


\subsubsection{Clustering of the Data}
To classify the data based on haptic fidelity and versatility, we divide each attribute into three categories. We therefore organize the dataset into an evenly split 3x3 grid, resulting in a total of 9 fields. The scale ranges from 0 to 4 on both axes.

For mid-fidelity haptic feedback, which is neither entirely abstract nor fully realistic, we introduce the term \textit{representational}. For medium versatility, which is neither highly specific to one use case nor entirely generic, we use the term \textit{broad}.

\subsubsection{Evaluating the Impact on Motor Learning}
\label{sec:impact_motor_learning}

To evaluate the impact of haptic feedback fidelity on motor learning for each study, we assign a score based on the findings provided:

\begin{itemize}
\item \textbf{++}: Studies showing a significant positive impact on motor learning, with all p-values indicating significance for the haptic feedback condition.
\item \textbf{+}: Studies showing a moderate positive impact on motor learning.
\item \textbf{o}: Studies showing no difference for haptic feedback on motor learning compared to other conditions, or studies showing positive as well as negative effects of haptic feedback on motor learning.
\item \textbf{-}: Studies showing a moderate negative impact on motor learning.
\item \textbf{-\hspace{-2mm} -}: Studies providing strong evidence that haptic feedback led to worse performance, with p-values indicating a significant negative impact on motor learning.
\end{itemize}

For some studies, the impact of haptic feedback on motor learning was not clear (e.g., no p-values were provided, or the haptic feedback condition was not considered in isolation). For these papers, we assumed an impact score on motor learning and marked this in the figure by adding a white mark in the center of the respective data points.
