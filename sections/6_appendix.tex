%%%%%% APPENDIX %%%%%%

\section*{Appendix}

\subsection{Queries}
\label{sec:queries}

\subsubsection{Scopus}
TITLE-ABS-KEY (motor AND (learning OR control OR training OR skills) AND (((virtual OR augmented) AND reality) OR ((remote OR virtual OR simulated) AND environment)) AND (((somatosensory OR haptic OR tactile OR proprioceptive OR kinesthetic OR cutaneous OR somatic) AND (cue* OR feedback OR rendering OR stimul*)) AND (fidelity OR realism OR accuracy OR precision OR exactness OR specificity)))

\subsubsection{IEEE Xplore}
motor AND (learning OR control OR training OR skills) AND (((virtual OR augmented) AND reality) OR ((remote OR virtual OR simulated) AND environment)) AND (((somatosensory OR haptic OR tactile OR proprioceptive OR kinesthetic OR cutaneous OR somatic) AND (cue* OR feedback OR rendering OR stimuli*)) AND (fidelity OR realism OR accuracy OR precision OR exactness OR specificity)) 

\subsubsection{PubMed}
motor AND (learning OR control OR training OR skills) AND (((virtual OR augmented) AND reality) OR ((remote OR virtual OR simulated) AND environment)) AND (((somatosensory OR haptic OR tactile OR proprioceptive OR kinesthetic OR cutaneous OR somatic) AND (cue OR cues OR feedback OR rendering OR stimuli*)) AND (fidelity OR realism OR accuracy OR precision OR exactness OR specificity))



\newpage
\begin{landscape}
\setlength{\tabcolsep}{2pt} % Default is 6pt;
\subsection{Overview Haptic Fidelity Scores}
\begin{table}[htbp]
\tiny
\begin{tabular}{lp{4cm}lllllllllllllllllllllcccc}
\multicolumn{1}{c}{\multirow{3}{*}{\textbf{Paper}}} &
  \multicolumn{1}{c}{\multirow{3}{*}{\textbf{Condition}}} &
  \multicolumn{15}{l}{\textbf{\small{Foundational Factors}}} &
  \multicolumn{7}{l}{\textbf{\small{Limiting Factors}}} &
  \multicolumn{1}{c}{\textbf{Feedback Fidelity}} &
  \multicolumn{1}{c}{\textbf{Versatiliy}} &
  \multicolumn{1}{c}{\textbf{Quality}} \\
\multicolumn{1}{c}{} &
  \multicolumn{1}{c}{} &
  \multicolumn{3}{c}{\textbf{BL}} &
  \multicolumn{1}{c}{\textbf{BA}} &
  \multicolumn{3}{c}{\textbf{St}} &
  \multicolumn{1}{c}{\textbf{Ma}} &
  \multicolumn{1}{c}{\textbf{SI}} &
  \multicolumn{3}{c}{\textbf{DoF}} &
  \multicolumn{1}{c}{\textbf{HP}} &
  \multicolumn{1}{c}{\textbf{SP}} &
  \multicolumn{1}{c}{\textbf{Sum}} &
  \multicolumn{1}{c}{\textbf{De}} &
  \multicolumn{1}{c}{\textbf{Di}} &
  \multicolumn{1}{c}{\textbf{HL}} &
  \multicolumn{1}{c}{\textbf{SE}} &
  \multicolumn{1}{c}{\textbf{Con}} &
  \multicolumn{1}{c}{\textbf{SL}} &
  \multicolumn{1}{c}{\textbf{Sum}} &
   &
   &
   \\
\multicolumn{1}{c}{} &
  \multicolumn{1}{c}{} &
  \multicolumn{1}{c}{Diff} &
  \multicolumn{1}{c}{Real} &
  \multicolumn{1}{c}{Total} &
  \multicolumn{1}{c}{} &
  \multicolumn{1}{c}{Diff} &
  \multicolumn{1}{c}{Real} &
  \multicolumn{1}{c}{Total} &
  \multicolumn{1}{c}{} &
  \multicolumn{1}{c}{} &
  \multicolumn{1}{c}{VR} &
  \multicolumn{1}{c}{Real} &
  \multicolumn{1}{c}{Total} &
  \multicolumn{1}{c}{} &
  \multicolumn{1}{c}{} &
  \multicolumn{1}{c}{} &
  \multicolumn{1}{c}{} &
  \multicolumn{1}{c}{} &
  \multicolumn{1}{c}{} &
  \multicolumn{1}{c}{} &
  \multicolumn{1}{c}{} &
  \multicolumn{1}{c}{} &
  \multicolumn{1}{c}{} &
   &
   &
   \\
\cite{Brickler2019} &
  - &
  0 &
  1 &
  4 &
  4 &
  1 &
  6 &
  3 &
  2 &
  3 &
  3 &
  3 &
  4 &
  4 &
  4 &
  3.50 &
  0 &
  0 &
  0 &
  0 &
  0 &
  0 &
  0.00 &
  3.50 &
  2 &
  1.00 \\
\cite{Chappell2022} &
  - &
  2 &
  4 &
  2 &
  1 &
  6 &
  9 &
  1 &
   &
  3 &
  8 &
  12 &
  3 &
  3 &
  4 &
  2.43 &
  2 &
  1 &
   &
  1 &
  0 &
   &
  6.00 &
  2.20 &
  2 &
  0.79 \\
\cite{Dai2023} &
  Physical: haptic feedback from physical slider &
  0 &
  4 &
  4 &
  4 &
  0 &
  6 &
  4 &
  4 &
  4 &
  2 &
  2 &
  4 &
  4 &
  4 &
  4.00 &
  0 &
  0 &
   &
  0 &
  3 &
  0 &
  9.00 &
  3.21 &
  1 &
  0.93 \\
\cite{Dai2023} &
  Dynamic: haptic feedback from physical slider and robotic arm &
  0 &
  4 &
  4 &
  4 &
  0 &
  6 &
  4 &
  4 &
  3 &
  2 &
  2 &
  4 &
  3 &
  4 &
  3.75 &
  0 &
  0 &
   &
  0 &
  3 &
  0 &
  9.00 &
  3.01 &
  2 &
  0.93 \\
\cite{Fehlberg2012} &
  Active Handrest with adaptive admittance strategy where   admittance gain was adjusted by time derivative of force input &
  1 &
  3 &
  3 &
  4 &
  2 &
  6 &
  3 &
  3 &
  3 &
  5 &
  6 &
  3 &
  4 &
  4 &
  3.38 &
  0 &
  1 &
   &
  0 &
  1 &
   &
  2.00 &
  3.34 &
  1 &
  0.86 \\
\cite{Grant2019} &
  - &
  0 &
  4 &
  4 &
  4 &
  4 &
  7 &
  2 &
  3 &
  3 &
  3 &
  3 &
  4 &
  4 &
   &
  3.43 &
  0 &
  1.5 &
  0 &
  0 &
  2 &
   &
  6.25 &
  3.09 &
  1 &
  0.86 \\
\cite{Gunter2022} &
  - &
  0 &
  2 &
  4 &
  0 &
  2 &
  7 &
  3 &
  3 &
  4 &
  12 &
  12 &
  4 &
  4 &
  4 &
  3.25 &
  0 &
  0 &
  0 &
  0 &
  0 &
   &
  0.00 &
  3.25 &
  2 &
  0.93 \\
\cite{Hanashima2023} &
  - &
  14 &
  18 &
  1 &
  1 &
  1 &
  1 &
  0 &
  1 &
  1 &
  46 &
  46 &
  4 &
   &
   &
  1.33 &
  0 &
  2 &
  1 &
  0 &
  0 &
   &
  5.00 &
  1.25 &
  3 &
  0.79 \\
\cite{Huang2006} &
  - &
  0 &
  4 &
  4 &
  4 &
  0 &
  5 &
  4 &
  4 &
  4 &
  1 &
  1 &
  4 &
  4 &
  4 &
  4.00 &
  0 &
  0 &
  0 &
  0 &
  0 &
  0 &
  0.00 &
  4.00 &
  0 &
  1.00 \\
\cite{Huang2007} &
  Vision-haptic &
  0 &
  3 &
  4 &
  4 &
  0 &
  4 &
  4 &
  4 &
  4 &
  1 &
  1 &
  4 &
  4 &
  4 &
  4.00 &
  0 &
  0 &
  0 &
  0 &
  0 &
  0 &
  0.00 &
  4.00 &
  1 &
  1.00 \\
\cite{Lee2012} &
  - &
  0 &
  2 &
  4 &
  4 &
  1 &
  4 &
  3 &
  2 &
  2 &
  12 &
  12 &
  4 &
   &
   &
  3.17 &
  0 &
  2 &
   &
  4 &
  0 &
   &
  20.00 &
  1.08 &
  1 &
  0.71 \\
\cite{LeeY2019} &
  - &
  3 &
  6 &
  2 &
  2 &
  5 &
  8 &
  2 &
  3 &
  3 &
  9 &
  9 &
  4 &
  4 &
  3 &
  2.88 &
  1.5 &
  0 &
  0 &
  0 &
  0 &
   &
  2.25 &
  2.84 &
  3 &
  0.93 \\
\cite{LiuH2019} &
  - &
  2 &
  8 &
  3 &
  3 &
  9 &
  9 &
  0 &
  1 &
  3 &
  6 &
  27 &
  1 &
  4 &
  4 &
  2.38 &
  2 &
  3 &
  0 &
  0 &
  0 &
  0 &
  13.00 &
  1.50 &
  3 &
  1.00 \\
\cite{Macuga2019} &
  - &
  0 &
  8 &
  4 &
  4 &
  0 &
  7 &
  4 &
  4 &
  3 &
  2 &
  2 &
  4 &
  3 &
   &
  3.71 &
  0 &
  0 &
  0 &
  0 &
  0 &
   &
  0.00 &
  3.71 &
  1 &
  0.86 \\
\cite{McAnally2023} &
  - &
  1 &
  4 &
  3 &
  2 &
  1 &
  1 &
  0 &
  2 &
  3 &
  6 &
  6 &
  4 &
  4 &
  4 &
  2.75 &
  0 &
  1 &
   &
  2 &
  0 &
  0 &
  5.00 &
  2.57 &
  4 &
  0.93 \\
\cite{Mohanty2023} &
  - &
  0 &
  12 &
  4 &
  4 &
  0 &
  9 &
  4 &
  4 &
  4 &
  12 &
  12 &
  4 &
  4 &
  2 &
  3.75 &
  0 &
  0 &
  0 &
  0 &
  0 &
  2 &
  4.00 &
  3.59 &
  1 &
  1.00 \\
\cite{Morris2007} &
  - &
  0 &
  3 &
  4 &
  0 &
  0 &
  4 &
  4 &
  3 &
  4 &
  3 &
  3 &
  4 &
   &
  4 &
  3.29 &
  0 &
  0 &
   &
  0 &
  0 &
  0 &
  0.00 &
  3.29 &
  2 &
  0.86 \\
\cite{Najdovski2020} &
  - &
  4 &
  6 &
  1 &
  3 &
  4 &
  8 &
  2 &
   &
  3 &
  1 &
  1 &
  4 &
   &
  4 &
  2.83 &
  2 &
  1 &
  0 &
  0 &
  2 &
  0 &
  9.00 &
  2.28 &
  1 &
  0.86 \\
\cite{Oezen2022} &
  Visuo-haptic without arm weight support &
  0 &
  4 &
  4 &
  4 &
  2 &
  5 &
  2 &
  2 &
  4 &
  6 &
  6 &
  4 &
   &
  4 &
  3.43 &
  0 &
  0 &
   &
  0 &
  1 &
  0 &
  1.00 &
  3.42 &
  3 &
  0.86 \\
\cite{Oezen2022} &
  Visuo-haptic with arm weight support &
  0 &
  4 &
  4 &
  4 &
  2 &
  5 &
  2 &
  1 &
  2 &
  6 &
  6 &
  4 &
   &
  4 &
  3.00 &
  0 &
  0 &
   &
  0 &
  1 &
  0 &
  1.00 &
  2.99 &
  3 &
  0.86 \\
\cite{Perez2023} &
  - &
  0 &
  4 &
  4 &
  4 &
  10 &
  12 &
  1 &
   &
  3 &
  2 &
  2 &
  4 &
   &
  4 &
  3.33 &
  2 &
  2 &
   &
  1 &
  3 &
  0 &
  18.00 &
  1.39 &
  2 &
  0.79 \\
\cite{Rodriguez2010} &
  - &
  0 &
  4 &
  4 &
  4 &
  0 &
  6 &
  4 &
  4 &
  3 &
  3 &
  6 &
  2 &
  4 &
  3 &
  3.50 &
  0 &
  0 &
   &
  0 &
  0 &
  0 &
  0.00 &
  3.50 &
  2 &
  0.93 \\
\cite{Trinitatova2023} &
  - &
  0 &
  6 &
  4 &
  3 &
  5 &
  9 &
  2 &
  3 &
  3 &
  8 &
  27 &
  1 &
   &
   &
  2.67 &
  0 &
  2 &
   &
  0 &
  1 &
  0 &
  5.00 &
  2.49 &
  3 &
  0.79 \\
\cite{Vaghela2021} &
  Active haptic feedback &
  0 &
  6 &
  4 &
  4 &
  3 &
  7 &
  2 &
  2 &
  3 &
  12 &
  12 &
  4 &
  4 &
   &
  3.29 &
  2 &
  2 &
  0 &
  0 &
  0 &
   &
  8.00 &
  2.76 &
  1 &
  0.86 \\
\cite{Vaghela2021} &
  Passive haptic feedback &
  0 &
  6 &
  4 &
  4 &
  1 &
  7 &
  3 &
  4 &
  4 &
  12 &
  12 &
  4 &
  4 &
   &
  3.86 &
  2 &
  1 &
   &
  0 &
  0 &
   &
  5.00 &
  3.61 &
  0 &
  0.79 \\
\cite{Vasudevan2020} &
  - &
  1 &
  4 &
  3 &
  2 &
  1 &
  1 &
  0 &
  2 &
  3 &
  6 &
  6 &
  4 &
  4 &
  4 &
  2.75 &
  0 &
  1 &
  0 &
  2 &
  0 &
  0 &
  5.00 &
  2.57 &
  4 &
  1.00 \\
\cite{Wall2000} &
  - &
  0 &
  2 &
  4 &
  4 &
  0 &
  4 &
  4 &
  4 &
  4 &
  3 &
  6 &
  2 &
  4 &
   &
  3.71 &
  0 &
  0 &
  0 &
  0 &
  0 &
   &
  0.00 &
  3.71 &
  2 &
  0.86 \\
\cite{Xia2023} &
  - &
  \multicolumn{1}{r}{-} &
  \multicolumn{1}{r}{-} &
  3 &
  3 &
  4 &
  5 &
  1 &
  2 &
  2 &
  40 &
  60 &
  3 &
   &
   &
  2.33 &
  1 &
  1 &
   &
  0 &
  0 &
   &
  2.00 &
  2.31 &
  3 &
  0.71 \\
\cite{Yang2023} &
  VR Controller only (vibrotactile) &
  0 &
  2 &
  4 &
  4 &
  6 &
  9 &
  1.00 &
  2 &
  3 &
  6.00 &
  6 &
  4 &
  3 &
  4 &
  3.13 &
  2 &
  2 &
   &
  0 &
  0 &
   &
  8.00 &
  2.63 &
  4 &
  0.86 \\
\cite{Yang2023} &
  VR Controller embedded in power tool grip (grip perception) &
  0 &
  2 &
  4 &
  4 &
  5 &
  9 &
  2.00 &
  3 &
  3 &
  6.00 &
  6 &
  4 &
  3 &
  4 &
  3.38 &
  2 &
  2 &
   &
  0 &
  0 &
   &
  8.00 &
  2.84 &
  2 &
  0.86 \\
\cite{Yang2023} &
  VR Controller embedded in power tool grip with battery and   head (grip, weight, shape) &
  0 &
  2 &
  4 &
  4 &
  3 &
  9 &
  3.00 &
  3 &
  3 &
  6.00 &
  6 &
  4 &
  3 &
  4 &
  3.50 &
  1 &
  2 &
   &
  0 &
  0 &
   &
  5.00 &
  3.27 &
  2 &
  0.86
\end{tabular}
\end{table}

\rule{\linewidth}{0.4pt} % Line with full text width and thickness of 0.4pt
\\ 

\small
\footnotemark[1]Here comes the text
\end{landscape}


\onecolumn
\tiny
\noindent
\footnotemark[1]Body Location (The degree to which the same location(s) on the body or body parts of the user are involved) \\
\footnotemark[2]Body Area (The degree to which the same extent of the body surface of the user is involved) \\
\footnotemark[3]Stimuli (the degree to which the same haptic receptors fo the user are involved) \\
\footnotemark[4]Magnitude (the degree to which the same intensity (e.g. force) and variation (e.g. texture) of stimuli are involved) \\
\footnotemark[5]Sensor integrity (the degree to which haptic stimuli and stimuli of other modalities match regarding the intent of the system) \\
\footnotemark[6]Degrees of freedom \\
\footnotemark[7]Hardware precision (the extent to which the hardware of the system can reproduce the detail of haptic feedback compared to the natural feedback) \\
\footnotemark[8]Software precision (the extent to which the system can calculate the detail of haptic feedback) \\
\footnotemark[9]Dependency \\
\footnotemark[10]Distinguishability \\
\footnotemark[11]Hardware latency \\
\footnotemark[12]Side Effects \\
\footnotemark[13]Constraints \\
\footnotemark[14]Software latency \\
\footnotemark[15]Addressed body parts that are different from the ones addressed in the natural occurrence of the task\\
\footnotemark[16]Addressed body parts in the natural occurrence of the task\\
\footnotemark[17]Degrees of freedom in VR\\
\footnotemark[18]Degrees of freedom in the natural occurrence of task\\
\footnotemark[19]See equation \ref{eq:diff}\\
\footnotemark[20]1kHz system, 240Hz screen \\
\footnotemark[21]Simulation not precise enough \\
\footnotemark[22]Serial connection with high speeds \\
\footnotemark[23]A bit less than the first condition, as it is not the slider itself, but a robotic arm with less movement speed (300mm/s) \\
\footnotemark[24]Intensity with 9N force is not quite matched \\
\footnotemark[25]400 dpi \\
\footnotemark[26]No weight nor vibration on the drill \\
\footnotemark[27]500 Hz frequency sufficient for this task \\
\footnotemark[28]Potentiometer for measuring ball position, an optical encoder with 3600 lines per revolution \\
\footnotemark[29]Frequency 1 kHz \\
\footnotemark[30]Optical quadrature encoder (2048 counts per revolution)	 \\
\footnotemark[31]Entire palm, ring- and little finger left out \\
\footnotemark[32]Texture is not matched but mimicked \\
\footnotemark[33]Mean errors 0.597mm in height, 3.74deg in roll, 0.919deg in pitch sufficient \\
\footnotemark[34]120 Hz refresh rate likely to be too little \\
\footnotemark[35]Texture is not matched, and intensity through vibration only to some extent \\
\footnotemark[36]Sensing error only about 1 degree \\
\footnotemark[37]Only one vibration for each finger cannot mimic distinguishable sensations when grasping the objects \\
\footnotemark[38]HMD with 640 x 480 pixels not very high  in resolution \\
\footnotemark[39]Body area way larger than in reality (palm vs. fingertip) \\
\footnotemark[40]Only vibration, no skin stretch possible when sliding on the test surface for example \\
\footnotemark[41]Noise in tracking head motion \\
\footnotemark[42]Tracking error of 3-5 degrees \\
\footnotemark[43]9 N force sufficient for trajectory, but not sufficient for the force that can be exerted by the user \\
\footnotemark[44]3 kHz refresh rate, 4k LED screen 60 Hz \\
\footnotemark[45]Difficult to quantify, but since the torso is the main part where a human would be affected by currents, we give this a medium-high score \\
\footnotemark[46]Difficult to quantify (human is not a submarine). But since the entire torso is covered with the haptic suit, we give this a medium-high score \\
\footnotemark[47]Only vibrational feedback provided  \\
\footnotemark[48]Cannot quite match the magnitude with vibrotactile feedback only \\
\footnotemark[49]VR DoF times 1.5, to get for DoF to a medium-high score. Medium score, since all the mechanoreceptors on the human torso, cannot be covered with 20 vibrators on the front and back, but they will imitate a solid impression of different mimicked sensations \\
\footnotemark[50]Good distinguishability due to sheer amount of vibrators, however only vibrations for each sensation \\
\footnotemark[51]Dedicated graphics card, and commercial controller should suffice for required precision \\
\footnotemark[52]Drill was not turning, thus no rotational inertia \\