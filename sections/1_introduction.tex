%%%%% INTRODUCTION %%%%%

\section{Introduction}
The virtual environment (VE) provides a comprehensive platform for users to acquire new skills and competencies, with applications ranging from physical therapy and rehabilitation to sports and industrial training, education, marketing, telerobotics, and research, as well as gaming and entertainment \cite{Wu2023TrainingReality, Oagaz2022PerformanceReality}. 
Furthermore, simulations in virtual reality (VR) enable researchers to study complex skills and assess the motor performance of users without giving up on experimental control. This approach facilitates the understanding of motor learning on a level that is not possible in the physical world \cite{Harris2021ExploringSimulator, Levac2019LearningReview}. 
Therefore it is not surprising that there is a great interest in VR-based training simulations, especially to teach skills that would otherwise have to be acquired in dangerous or sensitive environments, such as construction work, military training, or surgical operations \cite{Adami2021EffectivenessTeleoperation, Lele2013VirtualUtility, Qi2021VirtualScenario}.

During these physical interactions in the real world, for example, when learning how to operate a knee, we rely not only on visual feedback but also on information obtained through the mechanoreceptors in our skin, and proprioceptors in our tendons, joints, and muscles \cite{Gonzalez-Grandon2021ProprioceptionInteraction}. To provide these stimuli in the virtual world, tools or devices are required to reproduce the sensations we would expect when interacting with objects in the VE, which includes the perception of pressure, vibration, temperature, and the position and movement of body parts. Exposing the user to these sensations when they are interacting with objects in VR---generally denoted as somatosensory feedback---may offer a promising approach to increase the effectiveness of motor learning \cite{Sigrist2013AugmentedReview}. 
The degree to which the feedback in the VE mimics real-world interaction is determined by its fidelity \cite{Caird1996PersistentTraining}, which ranges from simple, binary cues such as a clicking sensation to high-fidelity feedback that feels as if the user is interacting with a real object \cite{Yang2023}.

% Added reviews, the second round of going to research gap
Many studies have explored the effect of \textit{augmented haptic feedback} and \textit{haptic guidance} on motor learning, which involves guiding the user through a movement. It has shown to be beneficial for initial learning, as it decreases the error and task completion time during training \cite{Caccianiga2021, LeeH2014, Fehlberg2012}, however, it can lead to overreliance, which may result in worsened long-term retention performance \cite{Oquendo2024}. 

In contrast, for \textit{haptic rendering}---which focuses on replicating the natural haptic sensations expected when interacting with virtual environments (VEs)---it remains unclear how varying levels of feedback fidelity influence motor learning.
While researchers previously assumed that higher fidelity correlates with better training performance in VR \cite{Caird1996PersistentTraining, Waller1998TheTraining}, as the interaction feels more natural, recent research suggests that there might be no linear correlation between the level of feedback fidelity and motor performance: Instead, motor performance decreased in conditions that yielded feedback with medium feedback fidelity, compared to scenarios with high and well-designed low feedback fidelity \cite{MahdiNabiyouni201520153DUI.}.

Since supplying high fidelity to a VE is very costly, it is important to determine under which circumstances low-fidelity feedback, such as somatosensory cues, is sufficient to increase the effectiveness of motor learning, and where high-fidelity feedback is necessary. 

In this work, we therefore systematically review current research to create an overview of the impact of different levels of somatosensory feedback fidelity on motor learning in VR. To create a common ground, we first evaluate each haptic feedback system in the studies utilizing the Haptic Fidelity Framework by Muender et al. \cite{Muender2022HapticReality}. We extend the framework by a confidence score, which stands for the quality of the paper, and evaluate the found impact on motor learning in the experiments. Then, we cluster the data and address similarities and differences in the studies, and how these influence the impact of the provided haptic feedback on motor learning. 
