%%%%% DISCUSSION %%%%%%

\section{Discussion}
% Summarize key findings of section results (\ref{sec:results}) in one or two paragraphs
% Interpretation of the results, discussion of the meaning in the context of motor learning
% Justify interpretation by citing the supporting key studies
% Discuss possible alternative interpretations of the data (if applicable)
% Mention limitations of my research 
    % Critique on framework: does not separate abstract from real tasks that are trained in the VE
    % Limitations of systematic review itself
    % Implications for future studies mention practical implications

\paragraph{\nameref{sec:abstractspecific}}
This cluster contains one paper only, in which the vibrational feedback has a negative impact on motor learning. Even though the rhythmic task itself would probably benefit from haptic feedback (as has been shown by \cite{Graham2008} for a similar task), the vibrational feedback does not match with the intended action by the user. As described by I. Lee et al, the vibration of the drumstick itself masked the vibrational feedback at higher frequencies, which made it harder for the participants to perceive the given feedback. Furthermore, the other types of feedback (e.g. FLOW, in which flowing bars across the screen showed when to hit the drum next) allowed for an anticipation of the next movement. This creates an uneven comparison with the vibrational feedback system. Therefore, the adverse effect on motor learning caused by poorly designed low-fidelity feedback, which is further disrupted by noise within the same modality, is unsurprising.

\paragraph{\nameref{sec:abstractbroad}}
The study conducted by Marchal-Crespo et al. used a self-built device with low haptic fidelity for haptic guidance. The task of bouncing a ball on a racket is classified as complex due to its nested redundancy: the number of execution variables exceeds the number of variables defining the task's outcome, both in execution (the joint configuration of the arm) and in the task itself (the positioning and speed of the racket when hitting the ball) \cite{Levac2019LearningReview}. Research on skill acquisition has shown how skill improvement is achieved by processing and exploiting variability in movement \cite{Sternad2018ItsLearning}. Due to the constraints of the haptic device, however, the movement was limited to the sagittal plane only, which might have hampered motor learning. Furthermore, the participants might have become overreliant on haptic guidance as the frequency of bouncing the ball increased. The beneficial effect of the haptic guidance on bouncing the ball at low frequencies, which resembles discrete movements, can be explained by the lower difficulty of the task itself, not requiring exploitation of movement variability for motor learning.

\paragraph{\nameref{sec:abstractgeneric}}
The study by Hanashima et al. had several limitations that may explain the lack of observed differences in feedback modalities on learning accuracy. Firstly, tactile feedback was provided only on the waist and head, whereas learning the postures involved the entire body, including the arms and legs. Secondly, vibrotactile feedback may not be suitable for teaching postures as it lacks directional information, making it hard to adjust movements accurately. Additionally, the 200 Hz vibration frequency targeted the Pacinian corpuscles, which adapt quickly, thus reducing the participant's sensitivity to vibration \cite{Hanashima2023}. Lastly, the study did not isolate haptic feedback alone, as visual feedback was always present, complicating the assessment of haptic feedback's effectiveness.

In conclusion, low-fidelity feedback does not seem to impact motor learning if it targets only a limited number of body parts involved in the task, lacks corrective information, and is designed in a way that reduces sensitivity.


\paragraph{\nameref{sec:representationalspecific}}
The positive impact of the haptic feedback system developed by Najdovski et al. on motor learning is unsurprising, as it provides haptic feedback that closely mimics the natural forces involved in tasks (e.g. forces experienced at the fingertips). Determining the stiffness of an object, which requires precise movements and online evaluation of tactile feedback, is facilitated by this system. The haptic feedback is delivered precisely where control of the object occurs, resulting in quicker transitions from feedback to reaction compared to translating visual cues into motion. This direct correspondence likely accounts for the faster evaluation of object stiffness and the higher accuracy in the experiments.


\paragraph{\nameref{sec:representationalbroad}}
The study by Chappell et al. demonstrates that mid-fidelity haptic feedback can enhance motor performance in a pick-and-place task compared to VR training without feedback. However, no significant improvement in skill transfer was observed when compared to a control group with no prior training. This outcome may be attributed to the small sample size (five participants per condition) and high inter-subject variability \cite{Chappell2022}. Additionally, discrepancies between the simulated and real hand limited the fidelity of the haptic feedback, potentially impacting the skill transfer results. Participants also reported high frustration levels with the haptic system, indicating a mismatch between expected and perceived stimuli and system-imposed constraints.

Gambaro et al. found that mid-fidelity feedback positively impacts motor learning when following a 2D trajectory. 
Similarly, Peña-Perez et al. noted that precise haptic feedback enhances tracking accuracy, as it is experienced when interacting with stiff virtual objects. However, interacting with softer objects, rendered with the same system, correlated with a worse motor performance. This effect may be explained by the fact that if we grasp a soft object, this also affects the perceived stimuli (e.g. a greater area of the fingers is covered, object deformation, etc.). This in turn may decrease the haptic fidelity of the system for the interaction with softer objects, which can negatively impact motor learning. The lack of vertical tactile feedback also potentially hindered accuracy in trajectory tracking.

In conclusion, mid-fidelity haptic feedback can positively influence motor learning by improving task performance and tracking accuracy. However, system constraints and a limited accuracy of the haptic feedback system can limit its effectiveness in improving overall accuracy and skill transfer.


\paragraph{\nameref{sec:representationalgeneric}}
The studies in this cluster can be divided into two groups: those using the HTC Vive controller \cite{Vasudevan2020, Yang2023, McAnally2023} and those utilizing wearable systems \cite{LiuH2019, Trinitatova2023, Xia2023}.

The studies using the controller have in common that the vibrational feedback provided by the controller aligns well with the tasks (e.g. the drilling task, which naturally includes vibrational feedback from the power tool \cite{Yang2023}). 
Vasudevan et al. and McAnally et al. conducted experiments with Fitts' tapping task. Providing vibrational feedback enhanced task performance, as the vibrational cues were sufficient for this simple task. The evaluation of these experiments also highlights a limitation of the haptic fidelity framework: due to the abstract nature of the task, there is no discrepancy between the experimental feedback and the natural occurrence of the task, resulting in a high haptic fidelity score, despite the feedback itself being abstract. The nature of the task (i.e. if it is abstract or if it has a natural counterpart) is not considered in Muender et al.'s framework.
Overall, the tasks involved in the studies of this cluster were well-designed and suited the vibrational feedback of the commercial controller.

The wearable systems are elaborate, featuring, for example, 40 vibration points on a vest \cite{Xia2023}, or a haptic glove with five vibration motors and a 3 DoF delta robot for haptic feedback in the palm \cite{Trinitatova2023}. These systems are well-designed to match the natural occurrence of tasks, such as grasping and lifting objects in VR with a glove-based system or telemanipulating a robot with hand movements. The tasks are much more complex, involving whole-body movements and 3D-trajectory following. As shown by Trinitatova et al and Yang et al., the haptic feedback was able to reduce mental demand, potentially enabling participants to focus more on task execution \cite{Trinitatova2023, Yang2023}.

In summary, mid-fidelity haptic feedback has a positive impact on motor learning, as it can reduce the mental load \cite{Yang2023}, improve task execution \cite{McAnally2023, LiuH2019} and increase accuracy \cite{Vasudevan2020}. However, the effectiveness of the feedback highly depends on the alignment with the task requirements. While simpler tasks seem to benefit from vibrational feedback, especially as the scale decreases so depth information can be obtained more easily through vibrational cues than through visual feedback only \cite{Vasudevan2020}, more complex tasks might require detailed and realistic feedback to enhance motor learning effectively.

\paragraph{\nameref{sec:realisticspecific}}
Four studies involve experiments that can be classified as abstract tasks: rolling a ball along a beam \cite{Huang2006}, grabbing a virtual knob and pointing it to a marker \cite{Dai2023}, tracing a target line in 2D \cite{Fehlberg2012}, and exciting a spring-inertia system \cite{Huang2007}. When evaluating the haptic feedback fidelity of systems, the difference between natural haptic feedback and system-provided feedback is often considered. For abstract tasks, this congruence is perfect since there is no natural occurrence of the task, thereby increasing the system's haptic fidelity.
While using abstract tasks to evaluate the impact of haptic feedback on motor learning may be valuable for research, these studies have limited value in assessing whether a condition helps learn real-world tasks.

While Macuga et al. and Mohanty et al. concluded that congruent haptic feedback positively affects motor learning, they also demonstrated the detrimental effects of incongruent haptic feedback \cite{Macuga2019, Mohanty2023}. These findings highlight yet another shortcoming of the haptic feedback fidelity framework, as it does not account for the direction of feedback, which can adversely affect motor learning.

For the peg-in-hole task, Mohanty et al. reported that subjects relied on visual information for coarse object manipulations but required kinesthetic perception for precise docking tasks. The disparity between the expected feedback (as seen on the display) and the provided feedback in this situation became especially salient and therefore may have led to the decrease in task performance \cite{Mohanty2023}. 
The human reliance on haptic feedback in fine manipulation tasks such as inserting a peg into a hole is logical, as some parts of the objects might be obscured by the objects themselves. This occlusion makes accurate object manipulation with visual feedback alone more challenging, thus increasing the participant's reliance on haptic cues.

The study by Fehlberg et al. is particularly relevant for evaluating the impact of haptic feedback fidelity, as it included six conditions with haptic feedback, five of which fall into this cluster. Due to small differences in the location and type of haptic feedback provided, the data points are closely grouped. The greatest improvement in tracking the 2D trajectory was achieved with a virtual spring fixture on the active handrest, a condition with a medium-high fidelity score compared to the other conditions involved in the experiment \cite{Fehlberg2012}. This performance increase may be explained by the fact that haptic guidance on the handrest via springs felt natural to the users while allowing for finer corrections with the fingers when using the Phantom Omni haptic device. The highest fidelity score in this cluster was achieved under the condition of using an active handrest and a virtual spring fixture on the Omni stylus. Participants performed only slightly worse, with a similar mean completion time. This higher fidelity score is attributable to the fixtures on the Omni stylus (i.e. the end-effector), which provide haptic feedback precisely at the point where task performance is measured, thereby enhancing the factor \textit{sensor integrity} \cite{Fehlberg2012}.

The study by S. C. Shamsunder and M. Manivannan, with a confidence score of $C=0.64$, lacked detailed experimental information and did not provide p-values or other statistical metrics. This limits the reliability and reproducibility of their findings and complicates comparisons with other studies. It highlights the importance of comprehensive reporting in haptic feedback research to ensure transparency, credibility, and the ability to draw meaningful conclusions.

In the study conducted by H. Lee and S. Choi, no significant long-term retention benefits from haptic feedback methods were found, as the effects did not extend beyond immediate post-training contexts \cite{LeeH2014}. We assume that this might be partially due to the task itself: it was neither very challenging, consisting of four curves on the same track for every trial, nor immersive, as participants only needed to steer the car without the need to use the gas or brake pedals or shifting gears. Additionally, participants were recruited without requiring a driving license, but they might have had prior experience with similar setups through video racing games \cite{LeeH2014}. These limitations hamper the informative value of this study.

Finally, the study conducted by Grant et al. demonstrates how simple haptic feedback can enhance motor learning \cite{Grant2019}. It led to smaller errors and the absence of overshoot. This suggests a better movement economy, as correcting undershoot requires less energy than correcting overshoot \cite{Grant2019}.
The researchers classify their audio-haptic feedback device as having low fidelity, as they provide simple vibration stimuli, compared to "high-end counterparts" that offer force feedback with higher resolution and refresh rates \cite{Grant2019}. However, after evaluating their device using Muender et al.'s haptic feedback fidelity framework, we found that the device indeed provides high-fidelity feedback in the context of the drilling task, as it fits well with the task and the 3D-printed handle adds additional realness. 
This example stresses the importance of a common framework for evaluating the fidelity of haptic feedback devices. Many researchers attribute varying levels of fidelity to their devices without providing detailed reasoning, complicating the differentiation of how different levels of fidelity impact motor learning. Despite some shortcomings, the framework by Muender et al. establishes a common ground for objectively comparing the effects of haptic feedback fidelity on motor learning \cite{Muender2022HapticReality}.

To conclude, congruent and realistic haptic feedback significantly enhances motor learning, particularly in tasks requiring fine manipulation, while incongruent feedback (especially when pointing in the opposite direction) can be detrimental.

%%% fidelity highly dependent on the task the device is used for %%%
\paragraph{\nameref{sec:realisticbroad}}

\textit{Haptic devices with stylus:} 
As can be seen from the great variety of tasks and the high haptic fidelity score, a stylus device offers a great possibility to provide haptic feedback to improve motor learning. Haptic feedback seems to be especially helpful when a third dimension is added to the task \cite{Feygin2002HapticSkill, Rodriguez2010, Gunter2022, Brickler2019}. It helps to optimally discriminate the depth component when following a 3D trajectory or to distinguish when a goal has been reached with the manipulator in VR \cite{Wall2000}. 
The study conducted by Wall et al. suggests that high-fidelity haptic feedback is especially helpful for ballistic, but not as helpful for non-ballistic movements. Ballistic movements, which are generally faster, might require our movement control to have a higher frequency. This higher frequency might be easier to achieve if the feedback is applied at the same location where it needs to be implemented, instead of having to visually detect the error, and then correcting the movement with the respective body part. Furthermore, for ballistic movements to be efficient, timing is essential to maintain momentum. Feygin et al. found that haptic feedback was especially helpful for the timing aspects of the task \cite{Feygin2002HapticSkill}, further supporting the hypothesis of haptic feedback being especially important for ballistic movements. 

\textit{3 DoF Delta haptic devices:}
The studies conducted by Gambaro et al. and Morris et al. show an opposite impact of haptic feedback on motor learning: While the effect of haptic feedback for the participants following a 2D-trajectory was positive when the feedback was applied once the users deviated from the assigned path \cite{Gambaro2014}, training with haptic feedback was significantly less effective than visual training when learning a force pattern \cite{Morris2007}. We assume that the reason for this difference lies in the way the haptic feedback was employed: In the experiment conducted by Morris et al., participants were asked to counteract opposing forces in the haptic training condition, keeping the handle in a stable position. Therefore, they needed to \textit{react} to the forces. 
When training under the visual condition, bars indicated the level of force relative to the maximum amount of force that had to be \textit{applied} at a certain time step. 
In the testing condition, participants also needed to \textit{apply} the correct amount of force (similar to the visual condition) at the right time instead of \textit{reacting} to opposing forces. 
This difference between the haptic training condition and the test condition may explain the poorer motor performance when participants trained with haptic feedback only \cite{Morris2007}.

\textit{DaVinci Research Kit:}
Caccianiga et al. highlighted improved accuracy and decreased displacement error for the groups that received augmented feedback \cite{Caccianiga2021}. However, the needed time for task completion did not change for any condition. This might be due to the time-accuracy trade-off: users improved their accuracy throughout the training, which required more time to complete the task. Consequently, any time improvement through the user's confidence might have been counterbalanced by additional time spent to minimize the error \cite{Caccianiga2021}. 
Oquendo et al. on the other hand, showed that the group undergoing haptic guidance had the worst performance for the testing on the final day, highlighting the effects of overreliance on haptic feedback: the participants became so dependent on the haptic guidance that this hindered performance once the assistive forces were removed \cite{Oquendo2024}. He also demonstrated how error amplification can foster motor learning, as improvement in performance even on the final day suggests that participants might have become better than the other groups although having more difficult conditions if the training was to go on for longer.


\textit{Self-built devices:} Chi et al., Dai et al., and Yang et al. demonstrate with their experiments how existing controllers can be adapted or new haptic devices can be built with relatively simple components to provide meaningful haptic feedback that is applicable to a broad range of use cases \cite{Chi2017, Dai2023, Yang2023}. The findings from the studies collectively highlight the positive impact of haptic feedback on motor learning, showing that higher fidelity enhances performance through smoother trajectories, which in turn suggest better movement economy \cite{Chi2017}, reduced errors, and decreased task completion times \cite{Dai2023, Yang2023}. 

The case of Yang et al., who compared three different haptic feedback fidelity conditions with each other, stresses the importance of a common ground in evaluating haptic feedback fidelity. Without giving further reasoning, they classified the conditions (i) VR controller only, (ii) VR controller embedded in power tool grip, and (iii) VR controller embedded in power tool grip with attached drill head into low (i), mid (ii), and high (iii) haptic feedback fidelity. Our analysis however has shown, that condition (i) offers representational (i.e. mid) fidelity, and the other two conditions can even be classified as having high haptic fidelity. Yang et al. found the participants to perform worse in the motor task for condition (ii), linking it to mid-fidelity and proposing an effect similar to the uncanny valley \cite{Mori2012TheValley, Yang2023}. Based on our evaluation, we cannot confirm the hypothesis that mid-fidelity feedback generally performs worse than low or high haptic feedback fidelity. 
However, when replicating a real-world task in VR, motor learning benefits from using a controller that is either an exact replica of the tool in VR (i.e. condition (iii)) or distinctly different (i.e. condition (i)). In Yang et al.'s experiment, a controller mimicking the weight of a power drill with an attached battery pack, but lacking the balance due to the missing drill head (ii), may disrupt the internal model we create of our arm dynamics and therefore negatively impact motor performance \cite{Yang2023}. 
Yang et al. however also showed, that condition (ii) reduced the perceived task complexity. This might be valuable for the motor learning performance of people with impaired motor skills \cite{Sigrist2013AugmentedReview}, aligning with the challenge point framework which states that less skilled subjects might not advance if the performed task is too demanding \cite{Guadagnoll2004ChallengeLearning}. 


\paragraph{\nameref{sec:realisticgeneric}}
Both studies undertaken by Vaghela et al. and Özen et al. depicted in figure \ref{fig:fidelity_plot} utilize high-fidelity feedback devices that provide extensive haptic feedback to users \cite{Vaghela2021, Oezen2022}. They demonstrate that combining high fidelity and high versatility is achievable but requires a well-designed and complex system with high sensor resolution that can provide stimuli that align well with the stimuli of the natural occurrence of the tasks but are still versatile enough to provide haptic feedback for other types of tasks.

It must be noted, that the experiment conducted by Lee et al. only involved 10 participants, of which two were associated with being more affected by the AA tracking motion, while three users were assigned to being haptic feedback-oriented users. This low sample size limits the generalizability of their findings. However, it might provide first evidence that the impact of haptic feedback on motor learning has a subjective component that is dependent on the users themselves.

Interestingly, Özen et al. showed that the group trained with arm weight support demonstrated worse transfer of learning in the long-term retention task compared to the group that trained without it. This suggests that while haptic feedback seems to enhance immediate task performance, supporting arm weight can significantly affect long-term skill retention. The support of the arm weight might have hindered retention, as there might have been a misalignment between the internal model of participants for their arm dynamics and their arm dynamics with weight support. This can lead to overshoot when moving in the vertical direction, which would explain the increase of vertical movement variability during training \cite{Oezen2022}. 

% Haptic feedback has been found to hinder motor learning, especially for tasks that naturally did not involve haptic feedback and when it was provided in a rather abstract way. Hanashima et al. for example provided vibrotactile feedback for posture learning and found the visual only condition to be superior over the visuo-tactile condition \cite{Hanashima2023}. However, when we learn a posture from another person in real life, we will also either depend on visual feedback only (for example when imitating the other person and using a mirror), or we will have actual guidance (for example from a physiotherapist). Vibrational feedback however does not seem to be a meaningful way to provide haptic feedback when learning postures.

% If the haptic rendering is amplified, but pointing in the same direction (for example when amplifying the errors, \cite{Oquendo2024}), this can be especially helpful for initially better skilled subjects. However, it is worse for subjects with initially less skill, as it might be overwhelming (according to challenge point theory). If the haptic feedback is incongruent, this is always worse for motor learning. For example, if the feedback is pointing in the opposite direction.

% \subsection{Shortcomings of framework}
% Framework mixes for haptic fidelity score multiple properties of the haptic devices and tries to accumulate these into one single score. This however does not cover the complexity of aspects that haptic devices have to offer. One example is the magnitude of feedback that is matched by the haptic device. This can either be explained by to which extent the haptic device can match the magnitude in real world, e.g. if the force in reality required 10N of force and the haptic device can render 10N of force. The other however is the force modality, as haptic devices can either render the natural forces of the object, but can also amplify the errors of the participant when doing a task, or have the forces point in the exact opposite direction. Even though this is only one in 8 evenly weighed foundational factors according to the framework \cite{Muender2022HapticReality}, it has a detrimental effect on motor learning. This is not appropriately taken into account in the framework.

\begin{comment}
Suggestions: For the post hoc evaluation of the haptic feedback fidelity, it would be helpful if researchers included specifications of the device used, such as hard- and software latency, maximum force, sensor and actuator accuracy, etc., even if the device is commercially available. This makes the objective rating of the feedback fidelity easier and quicker, as even for commercially used devices the specifications might not be available anymore at some point in the future.

Often, an artificial task is chosen, such as tracking a trajectory with a steering wheel, where the velocity of the car is constant to eliminate the use of foot pedals \cite{LeeH2014}. For the retention, the subjects are using the exact same setup, so parameters such as the degrees of freedom, and the stimuli of experiment and retention are the same, thus increasing the level of fidelity. However, the performance of the participants driving a real car, or even driving in a more realistic simulator is never evaluated, despite this actually being an interesting question. In the end, we want to know how we can effectively improve motor learning performance by providing haptic feedback to the user. If we are doing this only for artificial tasks, we are making the experimental setup easier and possibly cheaper, but not exploring the greater possibilities this technology might offer. 

One important finding is that haptic feedback might be especially important when learning a motor task that requires 3 dimensions. Studies that required the participants to follow a trajectory in 2D have often found no significant improvement in task performance when providing haptic feedback, compared to visual feedback only (\cite{Gambaro2014}, \cite{LiuG2014}, \cite{LeeH2014}), whereas studies involving trajectory following in 3 dimensions have shown a significant improvement with the introduction of haptic feedback (\cite{Grant2019}, \cite{Rodriguez2010}, ...)
\end{comment}