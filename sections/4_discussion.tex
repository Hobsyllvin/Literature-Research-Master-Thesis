%%%%% DISCUSSION %%%%%%
\pagebreak

\section{Discussion}
% Summarize key findings of section results (\ref{sec:results}) in one or two paragraphs
% Interpretation of the results, discussion of the meaning in the context of motor learning
% Justify interpretation by citing the supporting key studies
% Discuss possible alternative interpretations of the data (if applicable)
% Mention limitations of my research 
    % Critique on framework: does not separate abstract from real tasks that are trained in the VE
    % Limitations of systematic review itself
    % Implications for future studies, mention practical implications

\paragraph{\nameref{sec:specificabstract}}
This cluster contains one paper only, in which the vibrational feedback has a negative impact on motor learning. Even though the rhythmic task itself would probably benefit from haptic feedback (as has been shown by \cite{Graham2008} in a similar task), the vibrational feedback does not match with the intended action by the user, which is a unidirectional motion. Furthermore, as described by I. Lee et al, the vibration of the drumstick itself masked the vibrational feedback at higher frequencies, which made it harder for the participants to perceive the given feedback. Furthermore, the other types of feedback (e.g. FLOW, in which flowing bars across the screen showed when to hit the drum next) allowed for an anticipation of the next movement. This creates an uneven comparison with the vibrational feedback system. Therefore, the adverse effect on motor learning caused by poorly designed low-fidelity feedback, which is further disrupted by noise within the same modality, is unsurprising.


Haptic feedback has been found to hinder motor learning, especially for tasks that naturally did not involve haptic feedback and when it was provided in a rather abstract way. Hanashima et al. for example provided vibrotactile feedback for posture learning and found the visual only condition to be superior over the visuo-tactile condition \cite{Hanashima2023}. However, when we learn a posture from another person in real life, we will also either depend on visual feedback only (for example when imitating the other person and using a mirror), or we will have actual guidance (for example from a physiotherapist). Vibrational feedback however does not seem to be a meaningful way to provide haptic feedback when learning postures.

If the haptic rendering is amplified, but pointing in the same direction (for example when amplifying the errors, \cite{Oquendo2024}), this can be especially helpful for initially better skilled subjects. However, it is worse for subjects with initially less skill, as it might be overwhelming (according to challenge point theory). If the haptic feedback is incongruent, this is always worse for motor learning. For example if the feedback is pointing in the opposite direction.

\subsection{Shortcomings of framework}
Framework mixes for haptic fidelity score multiple properties of the haptic devices and tries to accumulate these into one single score. This however does not cover the complexity of aspects that haptic devices have to offer. One example is the magnitude of feedback that is matched by the haptic device. This can either be explained by to which extent the haptic device can match the magnitude in real world, e.g. if the force in reality required 10N of force and the haptic device can render 10N of force. The other however is the force modality, as haptic devices can either render the natural forces of the object, but can also amplify the errors of the participant when doing a task, or have the forces point in the exact opposite direction. Even though this is only one in 8 evenly weighed foundational factors according to the framework \cite{Muender2022HapticReality}, it has a detrimental effect on motor learning. This is not appropriately taken into account in the framework.

\begin{comment}
Suggestions: For the post hoc evaluation of the haptic feedback fidelity, it would be helpful if researchers included specifications of the device used, such as hard- and software latency, maximum force, sensor and actuator accuracy, etc., even if the device is commercially available. This makes the objective rating of the feedback fidelity easier and quicker, as even for commercially used devices the specifications might not be available anymore at some point in the future.

Often, an artificial task is chosen, such as tracking a trajectory with a steering wheel, where the velocity of the car is constant to eliminate the use of foot pedals \cite{LeeH2014}. For the retention, the subjects are using the exact same setup, so parameters such as the degrees of freedom, and the stimuli of experiment and retention are the same, thus increasing the level of fidelity. However, the performance of the participants driving a real car, or even driving in a more realistic simulator is never evaluated, despite this actually being an interesting question. In the end, we want to know how we can effectively improve motor learning performance by providing haptic feedback to the user. If we are doing this only for artificial tasks, we are making the experimental setup easier and possibly cheaper, but not exploring the greater possibilities this technology might offer. 

One important finding is that haptic feedback might be especially important when learning a motor task that requires 3 dimensions. Studies that required the participants to follow a trajectory in 2D have often found no significant improvement in task performance when providing haptic feedback, compared to visual feedback only (\cite{Gambaro2014}, \cite{LiuG2014}, \cite{LeeH2014}), whereas studies involving trajectory following in 3 dimensions have shown a significant improvement with the introduction of haptic feedback (\cite{Grant2019}, \cite{Rodriguez2010}, ...)
    
\end{comment}


\begin{comment}
\onecolumn

\begin{tikzpicture}[scale=3.6]
    
    % Add axis labels
    \foreach \x in {0,1,2,3,4} {
        \draw [very thin, lightgray](\x cm, 0-0.05) -- (\x cm, 4+0.05) node[anchor=north] {};
        \draw [very thin, lightgray](0-0.05,\x cm) -- (4+0.05,\x cm) node[anchor=east] {};
    }

    % Draw axes
    \draw[thick,<->] (0,2) -- (4,2) node[anchor=south west] {Haptic Fidelity};
    \draw[thick,<->] (2,0) -- (2,4) node[anchor=south] {Versatility};

    \node at (0,1.9) {\footnotesize{abstract}};
    \node at (4,1.9) {\footnotesize{realistic}};
    \node at (1.8,0.1) {\footnotesize{specific}};
    \node at (1.8,3.9) {\footnotesize{generic}};

    % Legend
    \draw[fill=white] (0.1,3.9) rectangle (1.2,3.25); % Legend border
    \node[anchor=west] at (0.15, 3.8) {\textbf{Legend}}; % Legend title
    
    \node[circle, fill=c1, inner sep=2.3pt] at (0.22, 3.65) {};
    \node[anchor=west] at (0.25, 3.65) {\footnotesize{Quality $q > 0.9$}};

    \node[circle, fill=c2, inner sep=2pt] at (0.22, 3.55) {};
    \node[anchor=west] at (0.25, 3.55) {\footnotesize{Quality $0.8 < q \leq 0.9$}};

    \node[circle, fill=c3, inner sep=1.7pt] at (0.22, 3.45) {};
    \node[anchor=west] at (0.25, 3.45) {\footnotesize{Quality $0.7 < q \leq 0.8$}};

    \node[circle, fill=c4, inner sep=1.4pt] at (0.22, 3.35) {};
    \node[anchor=west] at (0.25, 3.35) {\footnotesize{Quality $q \leq 0.7$}};
    
    % Sample data points
    \node[circle, fill=c1, inner sep=2.3pt] at (3.5,2) {};
    \node[circle, fill=c1, inner sep=2.3pt] at (3.22,2) {};
    \node[circle, fill=c1, inner sep=2.3pt] at (0.63,2) {};
    \node[circle, fill=c1, inner sep=2.3pt] at (3.38,2) {};
    \node[circle, fill=c1, inner sep=2.3pt] at (3.75,2) {};
    \node[circle, fill=c1, inner sep=2.3pt] at (2.63,2) {};
    \node[circle, fill=c1, inner sep=2.3pt] at (2.75,2) {};
    \node[circle, fill=c1, inner sep=2.3pt] at (3.21,0) {};
    \node[circle, fill=c1, inner sep=2.3pt] at (4,0) {};
    \node[circle, fill=c1, inner sep=2.3pt] at (4,1) {};
    \node[circle, fill=c1, inner sep=2.3pt] at (3.88,1) {};
    \node[circle, fill=c1, inner sep=2.3pt] at (1.5,3) {};
    \node[circle, fill=c1, inner sep=2.3pt] at (3.59,1) {};
    \node[circle, fill=c1, inner sep=2.3pt] at (3.09,2) {};
    \node[circle, fill=c1, inner sep=2.3pt] at (2.57,4) {};
    \node[circle, fill=c1, inner sep=2.3pt] at (3.21,1) {};
    \node[circle, fill=c1, inner sep=2.3pt] at (3.01,2) {};
    \node[circle, fill=c1, inner sep=2.3pt] at (3.25,2) {};
    \node[circle, fill=c1, inner sep=2.3pt] at (2.84,3) {};
    \node[circle, fill=c1, inner sep=2.3pt] at (3.7,1) {};
    \node[circle, fill=c1, inner sep=2.3pt] at (2.57,4) {};
    \node[circle, fill=c1, inner sep=2.3pt] at (3.5,2) {};
    \node[circle, fill=c2, inner sep=2.0pt] at (2.63,4) {};
    \node[circle, fill=c2, inner sep=2.0pt] at (2.84,2) {};
    \node[circle, fill=c2, inner sep=2.0pt] at (3.27,2) {};
    \node[circle, fill=c2, inner sep=2.0pt] at (3.34,1) {};
    \node[circle, fill=c2, inner sep=2.0pt] at (3.46,1) {};
    \node[circle, fill=c2, inner sep=2.0pt] at (3.46,1) {};
    \node[circle, fill=c2, inner sep=2.0pt] at (3.46,1) {};
    \node[circle, fill=c2, inner sep=2.0pt] at (3.62,1) {};
    \node[circle, fill=c2, inner sep=2.0pt] at (3.75,2) {};
    \node[circle, fill=c2, inner sep=2.0pt] at (3.09,1) {};
    \node[circle, fill=c2, inner sep=2.0pt] at (3.71,1) {};
    \node[circle, fill=c2, inner sep=2.0pt] at (3.29,2) {};
    \node[circle, fill=c2, inner sep=2.0pt] at (2.28,1) {};
    \node[circle, fill=c2, inner sep=2.0pt] at (3.42,3) {};
    \node[circle, fill=c2, inner sep=2.0pt] at (2.76,1) {};
    \node[circle, fill=c2, inner sep=2.0pt] at (3.71,2) {};
    \node[circle, fill=c3, inner sep=1.7pt] at (2.2,2) {};
    \node[circle, fill=c3, inner sep=1.7pt] at (2.8,2) {};
    \node[circle, fill=c3, inner sep=1.7pt] at (1.25,3) {};
    \node[circle, fill=c3, inner sep=1.7pt] at (1.39,2) {};
    \node[circle, fill=c3, inner sep=1.7pt] at (2.49,3) {};
    \node[circle, fill=c3, inner sep=1.7pt] at (3.61,0) {};
    \node[circle, fill=c3, inner sep=1.7pt] at (1.08,1) {};
    \node[circle, fill=c3, inner sep=1.7pt] at (2.31,3) {};
    \node[circle, fill=c4, inner sep=1.4pt] at (4,1) {};


    % Add citations to datapoints
    \node at (3.5,2.1) {\footnotesize{\cite{Brickler2019}}};
    \node at (3.22,2.1) {\footnotesize{\cite{Caccianiga2021}}};
    \node at (0.63,2.1) {\footnotesize{\cite{Crespo2015}}};
    \node at (3.38,1.9) {\footnotesize{\cite{Feygin2002HapticSkill}}};
    \node at (3.75,2.2) {\footnotesize{\cite{Feygin2002HapticSkill}}};
    \node at (2.63,2.1) {\footnotesize{\cite{Gambaro2014}}};
    \node at (2.75,1.9) {\footnotesize{\cite{Gambaro2014}}};
    \node at (3.21,0.1) {\footnotesize{\cite{Graham2008}}};
    \node at (4,0.1) {\footnotesize{\cite{Huang2006}}};
    \node at (4,1.1) {\footnotesize{\cite{Huang2007}}};
    \node at (3.88,1.1) {\footnotesize{\cite{LeeH2014}}};
    \node at (1.5,3.1) {\footnotesize{\cite{LiuH2019}}};
    \node at (3.59,1.1) {\footnotesize{\cite{Mohanty2023}}};
    \node at (3.09,1.9) {\footnotesize{\cite{Oquendo2024}}};
    \node at (2.57,3.9) {\footnotesize{\cite{Vasudevan2020}}};
    \node at (3.21,1.1) {\footnotesize{\cite{Dai2023}}};
    \node at (3.01,2.1) {\footnotesize{\cite{Dai2023}}};
    \node at (3.25,1.9) {\footnotesize{\cite{Gunter2022}}};
    \node at (2.84,3.1) {\footnotesize{\cite{LeeY2019}}};
    \node at (3.7,1.1) {\footnotesize{\cite{LiuG2014}}};
    \node at (2.57,4.1) {\footnotesize{\cite{McAnally2023}}};
    \node at (3.5,2.2) {\footnotesize{\cite{Rodriguez2010}}};
    \node at (2.7,4.1) {\footnotesize{\cite{Yang2023}}};
    \node at (2.84,2.2) {\footnotesize{\cite{Yang2023}}};
    \node at (3.27,2.2) {\footnotesize{\cite{Yang2023}}};
    \node at (3.34,1.1) {\footnotesize{\cite{Fehlberg2012}}};
    \node at (3.39,0.9) {\footnotesize{\cite{Fehlberg2012}}};
    \node at (3.51,0.9) {\footnotesize{\cite{Fehlberg2012}}};
    \node at (3.46,1.1) {\footnotesize{\cite{Fehlberg2012}}};
    \node at (3.62,0.9) {\footnotesize{\cite{Fehlberg2012}}};
    \node at (3.75,2.1) {\footnotesize{\cite{Fehlberg2012}}};
    \node at (3.09,1.1) {\footnotesize{\cite{Grant2019}}};
    \node at (3.71,1.2) {\footnotesize{\cite{Macuga2019}}};
    \node at (3.35,2.1) {\footnotesize{\cite{Morris2007}}};
    \node at (2.28,1.1) {\footnotesize{\cite{Najdovski2020}}};
    \node at (3.42,3.1) {\footnotesize{\cite{Oezen2022}}};
    \node at (2.76,1.1) {\footnotesize{\cite{Vaghela2021}}};
    \node at (3.71,1.9) {\footnotesize{\cite{Wall2000}}};
    \node at (2.2,2.1) {\footnotesize{\cite{Chappell2022}}};
    \node at (2.8,2.1) {\footnotesize{\cite{Chi2017}}};
    \node at (1.25,3.1) {\footnotesize{\cite{Hanashima2023}}};
    \node at (1.39,2.1) {\footnotesize{\cite{Perez2023}}};
    \node at (2.49,3.1) {\footnotesize{\cite{Trinitatova2023}}};
    \node at (3.61,0.1) {\footnotesize{\cite{Vaghela2021}}};
    \node at (1.08,1.1) {\footnotesize{\cite{Lee2012}}};
    \node at (2.31,3.1) {\footnotesize{\cite{Xia2023}}};
    \node at (4,0.9) {\footnotesize{\cite{Manivannan2008}}};

    
\end{tikzpicture}

\begin{table}[]
\begin{tabular}{@{}p{0.13\textwidth}p{0.08\textwidth}p{0.23\textwidth}p{0.19\textwidth}p{0.27\textwidth}@{}}
\toprule
 Feedback Fidelity & Reference & Experimental Design & Feedback Type & Findings \\ \midrule
 Low & \cite{Huang2007} &  &  & Auditory cues with feed-forward control do not help with task performance \\ \\
 Mid & \cite{Morris2007} & Follow trajectory with stylus & Haptic device with the opposite of the embedded force pattern & Performance was worse in haptic only compared to visual only, and far worse than visuo-haptic condition \\ \\
 High & \cite{Huang2007} & Maximum excitement of dynamic system with rotary handle at resonance frequency &  & decreased variance, increased precision \\ \bottomrule
\end{tabular}
\end{table}

\twocolumn
\end{comment}