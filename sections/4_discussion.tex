%%%%% CONCLUSION %%%%%
\section{Discussion}

\subsection{Low-Fidelity Haptic Feedback Caused by Flawed Experimental Designs Hinders Motor Learning}
According to the Haptic Fidelity Framework by Muender et al. \cite{Muender2022HapticReality}, the feedback fidelity score depends not only on the haptic device itself but also on how well it suits the designed task. In the investigated studies, a low haptic feedback fidelity score was achieved only by systems with design flaws or a mismatch between the feedback and the task, which increased the limiting factors in Muender et al.'s framework \cite{Muender2022HapticReality}, greatly reducing the fidelity score due to their squared impact (see equation \ref{eq:fidelity_score}). 

One example is noise or side effects of the device that overshadow the intended feedback. This does not only make it harder for the participants to perceive the given feedback, thus resulting in worse motor performance, but also greatly decreases the haptic fidelity score \cite{Lee2012}. Furthermore, an unfair comparison between feedback modalities may result in the impression of haptic feedback being disadvantageous for task execution: if the other types of feedback allow anticipation of the next movement, but the haptic feedback does not, a worse motor performance with haptic feedback seems unsurprising \cite{Lee2012}.

Research on skill acquisition has shown how skill improvement is achieved by processing and exploiting variability in movement \cite{Sternad2018ItsLearning}. Therefore, another design choice that decreases feedback fidelity and might hamper motor learning is constraining the user's movement in a way that hinders workspace exploration, which is part of the task in its natural occurrence.
Yet another shortcoming we could find is a misalignment between the body parts where feedback is provided, and the body parts involved in the expected movement: If vibrational feedback is provided on the waist and head for example, whereas learning the movement involves the entire body, this mismatch decreases the meaningfulness of the feedback, thus reducing motor performance \cite{Hanashima2023}.

In conclusion, studies with significant limitations---such as constraints, indistinguishable feedback, or feedback misaligned with the task—--achieved low haptic feedback fidelity, and generally reported negative or no impact of haptic feedback on motor learning.


\subsection{Mid- and High-Fidelity Haptic Feedback Aligning with Natural Forces Involved in the Task Improves Motor Learning}

In this review, all studies that achieved a medium feedback fidelity score and a majority of the studies with high-fidelity haptic feedback reported a positive impact of haptic feedback on motor learning, with some of them even noting significant improvements \cite{Yang2023, LiuH2019, Dai2023, Grant2019, Fehlberg2012, Huang2006}. 
Key positive outcomes included decreased response times \cite{Najdovski2020}, enhanced performance \cite{Chappell2022, Fehlberg2012, Mohanty2023}, higher accuracy \cite{Perez2023}, reduced mental demand \cite{Yang2023, Trinitatova2023} and better short-term retention \cite{LeeH2014}.

The systems involved in the study provide meaningful haptic feedback without major limitations that would increase the limiting factors in Muender et al.'s framework (see section \ref{sec:foundationallimiting}). Important is a good alignment between the natural forces involved in the tasks, and the feedback provided, which likely accounts for the reported faster task execution and higher accuracy \cite{Najdovski2020, Yang2023}. 

While motor performance for simpler tasks seems to benefit from vibrational feedback (e.g. for a drilling task \cite{Yang2023, Grant2019}), more complex tasks might require detailed and realistic feedback to enhance motor learning effectively (e.g. via glove-based systems \cite{Trinitatova2023} or haptic vests \cite{Xia2023}).

Haptic feedback seems to be especially helpful when a third dimension is added to the task \cite{Rodriguez2010, Gunter2022, Brickler2019}. It helps to optimally discriminate the depth component when following a 3D trajectory or to distinguish when a goal has been reached with the manipulator in VR \cite{Wall2000}.

It must be noted, however, that studies with the highest fidelity scores were often limited in versatility \cite{Huang2007, Vaghela2021}, as the haptic feedback system was precisely replicated in the virtual environment. While these systems offer great potential for studying the impact of realistic haptic feedback on motor learning in VR, the applicability of these systems to other types of tasks, as well as their meaningfulness in a broader context remain limited.
The highest versatility scores were achieved by studies with medium haptic feedback fidelity \cite{Yang2023, McAnally2023}, indicating promising future research possibilities as the feedback can be applied to a wide variety of tasks.

\subsection{High-Fidelity Feedback Enhances Motor Learning for Ballistic Movements}
% Maybe better for future research? As only supported by one paper

High-fidelity haptic feedback seems to be especially helpful for ballistic movements \cite{Wall2000}. As these are generally faster, they require our motor control systems to operate at a higher frequency to effectively manage the rapid initiation and execution of actions. This might be easier to achieve if the feedback is applied at the same location where the action needs to be implemented, instead of having to detect the error visually, and then correct the movement with the respective body part. 


\subsection{Incongruent Haptic Feedback Hinders Motor Learning}
If the haptic feedback is provided in a direction opposite of the feedback in the natural occurrence of the task---regardless of the level of haptic feedback fidelity---the haptic feedback seems to result in decreased motor performance. 

This was shown when participants felt reversed inertial cues when steering an electric vehicle, decreasing both their accuracy and efficiency \cite{Macuga2019}. 

Similarly, inconsistencies in haptic feedback between training and testing conditions can adversely affect motor learning. Exemplary is the study undertaken by \cite{Morris2007}, who had asked the participants in their experiment to \textit{counteract} opposing forces to stabilize a handle during training. However, the testing condition required participants to \textit{apply} the correct force proactively. This disparity may explain the poorer motor performance with haptic feedback training alone \cite{Morris2007}.

Lastly, as we create an internal model of our arm dynamics when interacting with our environment \cite{Pierella2019TheModels}, it seems to be helpful for motor learning if the haptic feedback provided by the device is either very abstract or very precise in replicating the haptic feedback we would experience when interacting with the real-world object. This was observed in the experiment by \cite{Yang2023}, as users performed best when using the controller only (abstract feedback), or when the power drill was exactly replicated. A controller that on one hand mimics the weight of a power drill with an attached battery pack, but on the other hand lacks balance due to a missing drill head, falls just short of the real-world object dynamics, and may therefore fall in the "uncanny valley of haptics" \cite{Berger2018TheHaptics}. This disrupts the internal model and therefore negatively impacts motor performance \cite{Yang2023}. 


\subsection{Study Limitations}

\subsubsection{Limitations of this Review}
There are several limitations regarding this systematic review.
Firstly, the evaluation of haptic feedback of the studies relied on partially subjective factors and estimations, especially for studies that had not reported certain metrics such as latency or precision of the haptic device used. This potential bias is further amplified by the fact that only one person conducted the evaluation.

Secondly, the decision to re-screen studies from 2022-2024 to account for recent advances in VR deviated from the strict PRISMA methodology, potentially reducing the reproducibility of the findings (see figure \ref{fig:prisma}).

Additionally, most studies in this review focused solely on motor performance during the experiments, without examining the transfer of skills to real-world tasks. Given that the primary goal of haptic feedback in VR is to enhance the (re-)learning of skills, the lack of data on skill transfer limits the relevance of these findings for real-life skill acquisition.


\subsubsection{Limitations of the Haptic Fidelity Framework}
The Haptic Fidelity Framework offers a great possibility to objectively rate a system's feedback fidelity, therefore creating a common ground to study the impact of different levels of feedback fidelity on motor learning. However, during this literature research, we noticed a few shortcomings of the framework that distort the impact of different levels of fidelity on motor learning and limit its meaningfulness.

\paragraph{Abstract Versus Realistic Tasks}
The fidelity of a feedback system is evaluated based on its correspondence to the natural occurrence of the task. However, for many abstract tasks, such as Fitts' tapping task \cite{Fitts1954TheMovement}, no natural occurrence exists, leading to artificially high haptic feedback fidelity scores. These tasks often allow for simpler haptic feedback (e.g., vibration on tapping), which also increases the versatility score of the haptic systems. 

Despite these high scores, the validity of these studies for meaningful motor learning tasks is limited. This creates an unfair comparison to feedback systems that offer both high versatility and high haptic feedback fidelity for complex motor learning tasks with natural occurrences \cite{LeeY2019, LiuH2019, Oezen2022}. These complex systems require intricate designs and may still not achieve the same fidelity and versatility scores as simpler designs used in abstract tasks.

\paragraph{Direction of Feedback}
The study of \cite{Macuga2019} demonstrated the detrimental effect of haptic feedback when it is directed opposite to the expected feedback. This significant impact on task performance is not adequately reflected in Muender et al.'s framework, as it is only addressed in the foundational factors (see section \ref{sec:foundationallimiting}), resulting in a minimal influence on the overall fidelity score. Similarly, \cite{Mohanty2023} found that incongruence between haptic and visual feedback had a substantial impact on motor learning. However, as it is addressed only under the foundational factor \textit{Sensor Integrity}, it has little effect on the feedback fidelity score \cite{Muender2022HapticReality}. This shortcoming in the framework distorts the relationship between feedback fidelity and its impact on motor learning.


\subsection{Practical Implications and Future Research}
In this systematic review we evaluated the impact of haptic feedback fidelity on motor learning in virtual reality (VR). Overall, the reviewed studies highlighted a generally positive impact of haptic feedback on motor learning, particularly for mid- and high-fidelity systems.

This review underscores the importance of using a common framework for evaluating the fidelity of haptic feedback systems to facilitate fair comparisons across studies. The Haptic Fidelity Framework by Muender et al., despite its shortcomings, offers a valuable tool for this purpose. We recommend its adoption by future researchers to ensure a comprehensive assessment of their haptic feedback systems. Furthermore, limitations of haptic feedback systems or tasks could be identified earlier if researchers assessed their systems with the framework during development, allowing for improvements in the experimental design.

Additionally, the study of skill transfer was notably rare among the assessed papers. This is unfortunate, as evaluating the long-term retention of motor skills under different conditions provides important insights into the actual impact of haptic feedback on motor learning. Incorporating retention tests into user studies would significantly enhance the meaningfulness of the research, displaying a potential research gap.

Another gap that may be addressed by future studies is to study and build systems that offer high versatility and realistic haptic fidelity. Even though these systems might be the most difficult to implement, especially for meaningful real-world tasks, they can provide great value for the study of the impact of haptic feedback on motor learning in VR and its real-world implications. 
